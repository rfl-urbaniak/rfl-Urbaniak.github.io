\documentclass[10pt, a4paper]{article}
%\usepackage{fontspec}


\linespread{1.2}


%bibstuff

\usepackage{natbib}


% DOCUMENT LAYOUT
\usepackage{geometry}

\geometry{top=5mm, bottom=12mm, left=2.7cm, right=2cm,marginparsep=2pt, marginparwidth=.8in}
\setlength\parindent{0in}



\usepackage{marginnote}
%\newcommand{\amper{}}{\chardef\amper="E0BD }
\newcommand{\years}[1]{\marginnote{\normalsize #1}}
\renewcommand*{\raggedleftmarginnote}{}
\setlength{\marginparsep}{7pt}
\reversemarginpar

% HEADINGS
%\usepackage{sectsty}
%\usepackage[normalem]{ulem}
%\sectionfont{\mdseries\upshape\Large}
%\subsectionfont{\mdseries\scshape\normalsize}
%\subsubsectionfont{\mdseries\upshape\large}


% PDF SETUP % ---- FILL IN HERE THE DOC TITLE AND AUTHOR
\usepackage[bookmarks, colorlinks, breaklinks,
pdftitle={Rafal Urbaniak - cv}, 	pdfauthor={Rafal Urbaniak},
	pdfproducer={http://nitens.org/taraborelli/cvtex}]{hyperref}
\hypersetup{linkcolor=blue,citecolor=blue,urlcolor=blue,filecolor=black}

\usepackage{sectsty}
\sectionfont{\vspace{-2mm} \large \sc}
\subsectionfont{\large \sc}


\usepackage{kpfonts}

\usepackage{enumitem}

%\usepackage{hyperref}
\usepackage{bibentry}

% DOCUMENT

\usepackage{url}




\begin{document}

\nobibliography{CVpapers}




\begin{center}
	
{\sc  \huge \textbf{Rafal Urbaniak}} \\
 \vspace{1mm}
\footnotesize  \href{mailto:rfl.urbaniak@gmail.com}{rfl.urbaniak@gmail.com} \Large $\cdot$ \footnotesize \href{https://rfl-urbaniak.github.io/}{rfl-urbaniak.github.io} \Large $\cdot$ \footnotesize 
 \href{https://www.linkedin.com/in/rfl-urbaniak/}{linkedin.com/in/rfl-urbaniak}

\end{center}

\vspace{-6mm}

\hrulefill 


\vspace{1mm}

\textbf{{\sc \Large Education}}\\


\vspace{-5mm}

\years{2008} PhD in Logic \& Philosophy of Mathematics, University of Calgary


\vspace{1mm}

\textbf{{\sc \Large Selected research grants \& awards}}\\

\vspace{-5mm}



\years{2022}   Bednarowski Trust Fellow,  University of Edinburgh 
\vspace{-2mm}\newline \scriptsize Research project: \emph{Explicating normic support}\normalsize


\years{2021}  Kosciuszko Foundation Fellow,  Northeastern University, Boston  
\vspace{-2mm} \newline  \scriptsize Research project:  
 \emph{Epistemological challenges to imprecise probabilism} \normalsize 


\years{2017-2021} Principal Investigator in a National Science Center project, University of Gdansk
\vspace{-2mm} \newline  \scriptsize Research project:  
 \emph{Conceptual, formal and practical
aspects of forensic and judicial applications of probabilistic tools} \normalsize 



\years{2012} Trinity College Long Room Hub Visiting Fellow, Trinity College Dublin
\vspace{-2mm} \newline  \scriptsize Research project:  
 \emph{ Mathematical existence, abstraction principles and real number theory} \normalsize 

\years{2011-2012} Visiting Fellow, Benares Hindu University, Varanasi
\vspace{-2mm} \newline  \scriptsize Research project:  
 \emph{Knowability paradox and Nyaya logic} \normalsize 


\years{2009} British Academy Visiting Fellow, Bristol University 
\vspace{-2mm} \newline  \scriptsize Research project:  
 \emph{Modal reconstructions of mathematical theories} \normalsize 

\vspace{1mm}

\textbf{{\sc \Large Positions held}}\\

\vspace{-5mm}


\years{2012-2019} Postdoctoral Fellow of Research Foundation Flanders


\years{2008-} Associate Professor, University of Gdansk (with academic leaves)



\years{2005-2008} TA, research assistant, instructor of record, University of Calgary





\vspace{2mm}	
\textbf{{\sc \Large Selected data science projects}}\\

\vspace{-6mm}

\begin{itemize} [leftmargin=*]
	
	
	
\item 	\href{https://rfl-urbaniak.github.io/LegalProbabilismBNs/}{\textbf{Legal Probabilism bayesian networks in R}} 
 \newline  \scriptsize \textbf{\textsf{R}} implementation  of \textbf{bayesian network methods} for criminal evidence evaluation, based on our work with  Marcello Di Bello on the \newline \href{https://plato.stanford.edu/entries/legal-probabilism/}{Legal  Probabilism entry} in the Stanford Encyclopedia of Philosophy


\normalsize 
\item 	\href{https://rfl-urbaniak.github.io/redditAttacks/}{
	\textbf{Short-term impact of personal attacks on Reddit user  activity}} 
\newline  \scriptsize  In cooperation with \href{https://www.samurailabs.ai/}{Samurai Labs}, we tracked 148,317 users  and identified personal attacks among 182,528 posts and comments  using their high precision software. I  analyzed the data  from three perspectives: (i) \textbf{classical statistical methods}, (ii) \textbf{Bayesian estimation}, and (iii) \textbf{model-theoretic analysis with hurdle and zero-inflated models}. They  agree: personal attacks decrease the victims' activity. \href{https://www.sciencedirect.com/science/article/abs/pii/S0747563221002958}{Results published} in \emph{Computers in Human Behavior}

\normalsize 
\item 	%\href{https://rfl-urbaniak.github.io/LegalProbabilismBNs/}{
\textbf{Bayesian estimation of multi-class bias in word2vec embeddings}%} 
\newline  \scriptsize We  propose \textbf{Markov chain Monte Carlo} methods  to supersede cosine-distance-based bias measures such as WEAT and argue that the resulting picture is not as clear as it initially might have seemed. 



\normalsize 
\item 	%\href{https://rfl-urbaniak.github.io/LegalProbabilismBNs/}{
\textbf{Probabilistic coherence measures over bayesian networks}%} 
\newline  \scriptsize Algorithms for calculating  main existing coherence measures over bayesian networks, with a new method essentially relying on the causal structure,  implemented in \textbf{\textsf{R}}, building on \textsf{bnlearn}, with application to multiple counterexamples to earlier proposals.




% \normalsize 
% \item %	\href{https://rfl-urbaniak.github.io/LegalProbabilismBNs/}{
% 	\textbf{Long-term impact of personal attacks on Reddit user  activity}%} 
% \newline  \scriptsize  Another cooperation with \href{https://www.samurailabs.ai/}{Samurai Labs}. Methods for \textbf{multiple time series with covariates} deployed to study  the impact of personal attacks on 25k users  of Reddit over 10 weeks.



\normalsize 
\item 	\href{https://rfl-urbaniak.github.io/backtesting/}{
	\textbf{MC in backtesting of optimized trading strategies}} 
\newline  \scriptsize  Implementation in \textbf{\textsf{R}} of \textbf{Monte Carlo methods} for gauging uncertainty in algorithmic trading strategy evaluation. Illustrates how correcting for multiple testing in optimization can undermine claims to significance.





\end{itemize}




\vspace{-1mm}	
\textbf{{\sc \Large Key skills}}\\

\vspace{-4mm}

\begin{tabular}{p{8cm}p{6cm}}
	\textbf{\textsf{R} tasks and tools} & \textbf{other} \\

\vspace{-6mm}
\begin{itemize}[leftmargin=*]   \setlength{\itemsep}{0pt} \scriptsize
	\item \normalsize Bayesian statistics	 \newline  \scriptsize \textsf{STAN, JAGS, BUGS, rethinking, rjags, runjags, BESTmcmc}
	
	\item \normalsize  Bayesian networks  \newline  \scriptsize \textsf{bnlearn, gRain}

	\item \normalsize  Discrete data models  \newline  \scriptsize
	 \textsf{ countreg,  vcd, vcdExtra, car, MASS}

	\item \normalsize  Time series, dynamic regression, algotrading  \newline  \scriptsize
	\textsf{xts, quantmod, tidyquant, TTR, fpp3}
	
	\item \normalsize  Data wrangling and visualization  \newline  \scriptsize \textsf{tidyverse, ggplot2, ggpubr}

	
	

\end{itemize}
&
\vspace{-6mm}
\begin{itemize}[leftmargin=*]   \setlength{\itemsep}{0pt}\scriptsize
	\item \normalsize Python \scriptsize
	\item \normalsize SQL \scriptsize 
	\item \normalsize Reproducible research \newline \scriptsize 
	R markdown, LaTeX, beamer,  Shiny
	\item \normalsize \textsf{Datacamp}: 24 courses completed \newline \scriptsize   130,102xp, 1,536 exercises aced 
\end{itemize}
\end{tabular}	


\vspace{-2mm}	
\textbf{{\sc \Large Academic activities}}\\

\vspace{-6mm}

\begin{itemize}[leftmargin=*]
	 \setlength{\itemsep}{0pt}\scriptsize
	\item \normalsize  \textbf{Multiple university courses taught} related to probability, statistical programming and inference
	\newline
	\scriptsize e.g.: statistical methods in criminology; probability \& philosophy of juridical evidence evaluation; uncertainty and \textbf{\textsf{R}} programming; interpretations of probability; decisions, games, and social choice theory; cognitive failures; meta-arithmetic; formal theories of truth; ontology of mathematics, logic I, logic II
	
	\item \normalsize  \textbf{3 books and ca.\!  30 research papers} in top academic journals\newline
	\scriptsize e.g.: Stanford Encyclopedia of Philosophy; Computers in Human Behavior; Aggressive Behavior; Review of Symbolic Logic; Law, Probability \& Risk; The International Journal of Evidence \& Proof, Journal of Applied Logics, Artificial Intelligence and Law 
	
	\item \normalsize \textbf{30 invited lectures}
	 \newline
	 \scriptsize e.g.: University of Oxford, Paris I Panth\'eon-Sorbonne, University of Edinburgh,   Kyoto University, Keio University, Nagoya University,  University of Turin
	 
	 \vspace{-2mm}	  
	\scriptsize 
	 \item \normalsize  \textbf{13 major research grants and awards}, 15  international conferences organized,   referee for 18  journals
	
	
	\vspace{-2mm} \scriptsize 
	\item \normalsize \textbf{Research group coordinator} of \href{http://lopsegdansk.blogspot.com/p/lopse-team.html}{LoPSE research group} (4-7 researchers) since 2013
		
	
	
\end{itemize}
\thispagestyle{empty}





\pagebreak 

% \large {\sc \textbf{Research projects \& stays}}\normalsize


% %\vspace{-1mm}

% \years{2022 \\ \scriptsize (awarded)} \emph{Explicating normic support}

% Bednarowski Trust Fellow, host: Martin Smith, University of Edinburgh, Scotland


% \years{2021 \\ \scriptsize (awarded)}   \emph{Epistemological challenges to imprecise probabilism}

%  The Kosciuszko Foundation Fellowship, host: Branden Fitelson, Northeastern Unversity, USA

% \years{2017-} \emph{Conceptual, formal and practical
% aspects of forensic and judicial applications of probabilistic tools}

% Supervision of one post-doctoral and two pre-doctoral researchers

%  Sonata Bis  grant of the Polish Center for Science, Poland %(ca. 220 000 EUR)


% \years{2015-2019} \emph{Forensic and judicial probability vs. probability theory. Formal tools for real-life justifications}

%  Postdoctoral Fellowship of the Research Foundation Flanders, Belgium  %(ca. 100 000 EUR)


% \years{2012} \emph{Mathematical existence, abstraction principles and real number theory}

% Trinity College Long Room Hub Visiting Fellow, Trinity College Dublin, Ireland


% \years{2012-2015} \emph{Mathematical theories with parsimonious ontologies: their strengths and limitations}


% Postdoctoral Fellowship of the Research Foundation Flanders, Belgium %(ca. 80 000 EUR)

% \years{2011-2012} \emph{Knowability paradox and Nyaya logic}

% Visiting Fellow, Benares Hindu University, Varanasi, India


% \years{2009}  \emph{Modal reconstructions of mathematical theories}

% British Academy Visiting Fellow, Bristol University, United Kingdom


%  \years{2009}  \emph{Development of adaptive logics for the study of central topics in contemporary philosophy of science}

%  Postdoctoral Researcher, Centre for Logic and Philosophy of Science, Ghent University, Belgium


% \years{2008} \emph{Contextual and formal-logical approach to scientific problem solving processes}

%  Postdoctoral Researcher, Centre for Logic
% and Philosophy of Science, Ghent University, Belgium

% \years{2005-2006}  \emph{The History of Logical Metatheory}

%  Research Assistant to Prof. Richard Zach, University of Calgary, Canada

% \vspace{2mm}



\textbf{{\sc \Large Selected journal papers}}\\

\vspace{-4mm}




\setlength{\emergencystretch}{3em}
%\small

%Leuridan

\years{2022}
 
\bibentry{Urbaniak2022personal}


\vspace{0.5mm}

\years{2021}

\bibentry{Urbaniak2021Informal}

\vspace{0.5mm}

\bibentry{Bilewicz2021hate}


\vspace{0.5mm}

\bibentry{sep-legal-probabilism}
%\bibentry{Urbaniak2020a}

\vspace{0.5mm}

\bibentry{Urbaniak2020}

\vspace{0.5mm}


\years{2020}


\bibentry{Urbaniak2020RiskBased}

\vspace{0.5mm}


\years{2019}

%\renewcommand\baselinestretch{1}


\bibentry{urbaniak2019ProbabilisticModelsLegal}


\vspace{0.5mm}



\bibentry{Urbaniak2019standards2}

\vspace{0.5mm}

\years{2018}

\bibentry{Urbaniak2017Narration-in-ju}


\vspace{0.5mm}

\bibentry{Stefaniak2019}

\vspace{0.5mm}

 \bibentry{Urbaniak2017Many-valued-log}


\vspace{0.5mm}

\years{2017}
\bibentry{Urbaniak2018ambiguity}










\vspace{0.5mm}


%\bibentry{Urbaniak2017Reconciling-bay}



\years{2016}

\bibentry{Urbaniak2017Challenging-Lew}

\vspace{0.5mm}

\bibentry{Urbaniak2016LesniewskiStyle}


\vspace{0.5mm}


%\pagebreak 

 \years{2015} \bibentry{Urbaniak2014a}

\vspace{0.5mm}

 %inapplicability of paraconsistent
\years{2014} \bibentry{Urbaniak2014inapplicability}

\vspace{0.5mm}

\bibentry{Urbaniak2014_generalized}


% \vspace{0.5mm}

%  \bibentry{Urbaniak2014}


\vspace{0.5mm}



\years{2013}\nopagebreak \bibentry{CieslinskiUrbaniak2013}



\vspace{0.5mm}

% paper in ERK on nubers and propositions
\years{2012} \bibentry{Urbaniak2012-URBNAP}


\vspace{0.5mm}

%platonic
\bibentry{Urbaniak2011c}


\vspace{0.5mm}

%\bibentry{Urbaniak2009a}


\vspace{0.5mm}



\bibentry{Urbaniak2011b}


\vspace{0.5mm}

%\pagebreak

\years{2011}\bibentry{Urbaniak2011}




\vspace{0.5mm}


 \years{2010} \bibentry{Urbaniak2010a}



\vspace{0.5mm}

\bibentry{Urbaniak2010c}

\vspace{0.5mm}


\bibentry{Urbaniak2009}


\vspace{0.5mm}

\years{2009}\bibentry{Urbaniak2009e}


\vspace{0.5mm}


\bibentry{urba:note09}


\vspace{0.5mm}

%\bibentry{Urbaniak2009d}

%\bibentry{Urbaniak2009c}
%\bibentry{Urbaniak2009b}


\years{2008}
\bibentry{Urbaniak2008:paradox}


\vspace{0.5mm}

\bibentry{Urbaniak2008a}


\vspace{0.5mm}

\years{2007}
\bibentry{Urbaniak2007}

\vspace{0.5mm}

\years{2006}
\bibentry{RafalUrbaniak08012006}


\vspace{0.5mm}

\bibentry{urbaniak06:_some_non_stand_inter_of}


\vspace{0.5mm}


 \bibentry{urbaniak06:_ontol_funct_of_lesniew_elemen_ontol}







% \years{2009}
%\bibentry{Urbaniak2009g}


%\bibentry{Urbaniak2009f}


%\bibentry{Urbaniak2009i}


%\bibentry{Urbaniak2009h}

%\pagebreak


%\years{2008}\bibentry{Urbaniak2008a}


%\years{2007} \bibentry{Urbaniak2007}

 %\pagebreak

%\years{2003} \bibentry{Urbaniak2003}

\vspace{2mm}



\vspace{2mm}


\textbf{{\sc \Large Books published}}\\

\vspace{-4mm}



\years{2017} \bibentry{Urbaniak2017The-Road-Less-T}


\years{2016} \bibentry{Urbaniak2016Slightly-irresp}



\years{2013} \bibentry{Urbaniak2013Lesniewskis-sys}

\vspace{2mm}


\vspace{2mm}


\textbf{{\sc \Large Other selected publications}}\\

\vspace{-4mm}



%\setlength{\emergencystretch}{3em}



%\vspace{0.5mm}

\years{2019}
\bibentry{Urbaniak2019reals}

\vspace{0.5mm}

\bibentry{GodziszewskiUrbaniak2019Liar}


\vspace{0.5mm}

\years{2018}

\bibentry{urbaniak2018induction}





\vspace{0.5mm}

\bibentry{Urbaniak2017_survey_provability}


\vspace{0.5mm}


\bibentry{UrbaniakGodziszewski2018}



\vspace{0.5mm}


\years{2017}

\bibentry{Urbaniak2017_Reconciling}


\vspace{0.5mm}

\bibentry{Urbaniak2017_beating}


\vspace{0.5mm}

%\years{2016}

%\bibentry{Urbaniak2016Mathematical-Ex}





\vspace{0.5mm}

\years{2011}
\bibentry{Urbaniak_Rostalska2011}




\vspace{2mm}

\pagebreak


\large {\sc \textbf{Selected talks}}\normalsize

\setlength{\emergencystretch}{3em}
%\small


%Leuridan

\vspace{1mm}


 {\sc \textbf{Invited and departmental talks}}\normalsize

\noindent

\vspace{1.5mm}

\years{2021}

\emph{Story coherence with Bayesian Networks} with Alicja Kowalewska, invited webinar at Center for Logic, Language, and Cognition University of Turin, Italy


\emph{Story coherence with Bayesian Networks}, invited graduate class taught with Alicja Kowalewska at Arizona State University, USA



\years{2018}

\emph{Potential Infinity, Le\' sniewskian Definitions, Arithmetic \& Yablo sequences}, Logic, Language, and Ontology -- WaragaiFest 2018, Tokyo, Japan

\vspace{0.5mm}


\emph{Modal Quantifiers, Potential Infinity, and Yablo sequences}, Proof Society Summer School 2018, Ghent University, Belgium

\vspace{0.5mm}

\years{2017} \emph{Narrations in judiciary fact-finding
and the difficulty about conjunction}, Faculty of Philosophy, University of Groningen, Netherlands

\vspace{0.5mm}

\emph{Paradoxes of informal provability and many-valued indeterministic provability logic}, Kyoto Philosophical Logic Workshop III, Kyoto University, Japan

\vspace{0.5mm}


\emph{Narration in Judiciary Fact-Finding: A Probabilistic Explication}, Tokyo Forum for Analytic Philosophy, The University of Tokyo, Japan


\vspace{0.5mm}


\years{2015}  \emph{Non-deterministic logic(s) of (informal) provability,}  University of Oxford (United Kingdom), University of Oslo (Norway)


\vspace{0.5mm}


 \emph{A non-classical logic of informal provability} (with Pawel Pawlowski), Munich Center for Mathematical Philosophy, Germany


\vspace{0.5mm}


\emph{Non-deterministic logics of provability}, $\exists$ntia et Nomin$\forall$ 2015, Cracov, Poland


\vspace{0.5mm}


  \emph{Workshop on  probabilistic logic}, Metafizz Student Club,
Trinity College Dublin, Ireland


\vspace{0.5mm}



\years{2013} \emph{Mereology and mathematics. The case of Le\'sniewski and his continuators}, Parts in Logic and Metaphysics, Universit{\'e} de Rennes I, France


\vspace{0.5mm}


\emph{Stanis\l aw Le\' sniewski: Re-thinking the philosophy of mathematics}, Academia Europaea 25th anniversary congress, Wroc\l aw, Poland


\vspace{0.5mm}




\years{2012} \emph{G{\"o}delizing the Yablo sequence},  Trinity College Dublin, Ireland


\vspace{0.5mm}



\emph{Neologicism: for real(s)}, Numbers and Truth, University of Gothenburg, Sweden



\vspace{0.5mm}

 \emph{Abstraction without abstracta}, Polish Philosophical Congress, Wis{\l}a, Poland


\vspace{0.5mm}


\emph{The philosophical usefulness of paradoxes},  Banaras Hindu University, Varanasi, India



\vspace{0.5mm}

 \emph{Busting a myth about Le\' sniewski and definitions},  Warsaw University, Poland


\vspace{0.5mm}





\years{2010} \emph{Frames, similarity, many-valuedness}, Workshop on Types and Frames,    Heinrich-Heine Universit\"at, D\"usseldorf, Germany



\vspace{0.5mm}


\emph{Yellow Card for Salmon},   Oxford University, United Kingdom


\vspace{0.5mm}


  \emph{Swinburne's modal argument for the existence of the soul} (with Agnieszka Rostalska),  Canterbury University, Christchurch, New Zealand



\vspace{0.5mm}



\years{2009} \emph{Modalities+Identities-Entities}, Structure and Identity and Abstraction Workshop,      Bristol University, United Kingdom



\vspace{0.5mm}

\emph{Galileo on falling bodies, Platonism, Dynamic Reasoning}, Universit\'e  Paris I Panth\'eon-Sorbonne, France





\vspace{0.5mm}

 \emph{Nominalist   Neologicism},  University of St. Andrews, Scotland


\vspace{0.5mm}

\emph{Abstraction without Abstracta},    Trinity College Dublin, Ireland


\vspace{0.5mm}


 \emph{Modality and Existence}, keynote address, Polish Society for Logic and Philosophy of Science annual \newline meeting,  Warsaw University, Poland


\vspace{0.5mm}



 \emph{Logic and Religion: a case study},    Bristol University, United Kingdom



\vspace{0.5mm}


\emph{Nameability and Mathematical Existence},  Logic \& Language Seminar,   University of Edinburgh, Scotland


\vspace{0.5mm}

\years{2008}
  (1) \emph{What   sentences   refer   to} and (2)
\emph{Quantified propositional modal  logic and philosophy} (with Agnieszka Rostalska),       University   of    Edinburgh,
Scotland


\vspace{0.5mm}


\years{2006} \emph{Demonstratives in Logica Magna}, Workshop on Medieval Nominalism,  University of Quebec, Montreal, Canada


\vspace{0.5mm}



 \years{2005}\emph{On ontological functors of Le{\'s}niewski's Ontology},  Polish Association for Logic and Philosophy of
Science (award reception),  Warsaw University, Poland

\vspace{1.5mm}

\pagebreak 

 {\sc \textbf{Conference \& workshop presentations}}\normalsize



\noindent

\vspace{1.5mm}


\years{2020}  (1) \emph{Story coherence with Bayesian Networks} with Alicja Kowalewska, and (2) \emph{ Imprecise Credences Can Increase Accuracy wrt. Claims about Expected Frequencies}, Bayesian Epistemology: Perspectives and Challenges, LMU, Munich, Germany





\years{2019} (1) \emph{Modal Quantifiers, Potential Infinity, and Yablo sequences} with Micha\l\, Tomasz Godziszewski, and (2) \emph{Combining truth values with provability values: a non-deterministic logic of informal provability} with Pawe\l\, Pawlowski, International Congress of Logic, Methodology and Philosophy of Science and Technology, Prague, Czech Republic


\years{2018}

\emph{Infinite liar in a (Modal) Finitistic Setting}, 8th Indian Conference on Logic and its Applications, ICLA 2019, Delhi, India

\vspace{0.5mm}

(1) \emph{Abstraction Principles via Le{\'s}niewskian definitions. Potential infinity and arithmetic} and (2) \emph{Narrations in judiciary fact-finding and the difficulty about conjunction}, Unilog 2018, Vichy, France

\years{2017} \emph{Reconciling Bayesian epistemology and narration-based approaches to judiciary fact-finding},   Sixteenth conference on
Theoretical Aspects of Rationality and Knowledge 2017, University of Liverpool, United Kingdom


\vspace{0.5mm}


\emph{Beating the Gatecrasher Paradox with Judiciary Narratives},
Logic, Rationality, and Interaction 2017: 6th International Workshop, Sapporo University, Japan



\vspace{0.5mm}


 \emph{Probabilistic Explication of the Notion of Narration as Used in Judiciary Contexts}, Progic 2017: The 8th workshop on Combining Probability and Logic,  Munich, Germany



\vspace{0.5mm}

\emph{Probabilistic explication of the notion of narration as used in judiciary contexts}, $\exists$ntia et Nomin$\forall$ 2017,  Palolem, India


\vspace{0.5mm}


\years{2016} \emph{Understanding conditionals}, Hokudai-Ghent Joint Symposium “Connecting Japan and Belgium”,  Ghent, Belgium


\vspace{0.5mm}


\emph{Failxplication of truth (untyped)}, Language and metalanguage, logic and meta-logic. Revisiting Tarski's hierarchy,  Louvain-la-Neuve, Belgium


\vspace{0.5mm}

\emph{Probabilistic explication of the notion of narration as used in judiciary contexts}, Justification of court decisions: law-theoretic and interdisciplinary perspectives,  Jab{\l}onna, Poland


\vspace{0.5mm}


\emph{Not too abstract abstraction principles for real numbers}, Polish Ontology Nowadays,  Warsaw, Poland


\vspace{0.5mm}

 \years{2015} \emph{The non-applicability of selected paraconsistent logics}, 15th Congress of Logic, Methodology and Philosophy of Science,  Helsinki, Finland



\vspace{0.5mm}

 \emph{First steps towards non-classical logic of informal provability} (with Pawel Pawlowski), 15th Congress of Logic, Methodology and Philosophy of Science,  Helsinki, Finland

% \emph{The non-applicability of selected paraconsistent logics}, 5th World Congress on Universal Logic,  Istanbul, Turkey

% \emph{First steps towards non-classical logic of informal provability} (with Pawel Pawlowski), 5th World Congress on Universal Logic,  Istanbul, Turkey



\vspace{0.5mm}


\years{2014} \emph{Verbality and Disagreement in the Nyaya-Samkhya debate about causality}, Udaharanas, Drstantas and Nyayas in Texts and Contexts,  Gujarat University, Ahmendabad, India



\vspace{0.5mm}

 \emph{Irrational bookies}, Third Reasoning Club Conference, University of Kent, United Kingdom


\vspace{0.5mm}


 \emph{A comparative study of selected filtered paraconsistent logics}, 5th World Congress on Paraconsistency,  Kolkata, India


\vspace{0.5mm}



%\years{2013} \emph{Irrational bookies}, Entia et Nomina 2013,   University of Gda\'nsk,  Poland



\years{2012} \emph{G{\"o}delizing the Yablo sequence}, Trends in Logic XI -- Advances in Philosophical Logic,  Ruhr-Uni\-ver\-si\-t{\"a}t Bochum, Germany


\vspace{0.5mm}



\years{2011} \emph{Haecceitas and possible worlds}, Philosophical perspectives on individuals,  University of Gda\'nsk, Poland


\vspace{0.5mm}






(1) \emph{Busting a myth about Le\' sniewski and definitions} and (2) \emph{Platonic thought experiments? How on earth?}, 14th Congress of Logic, Methodology and Philosophy of Science,  Nancy, France


\vspace{0.5mm}


(1) \emph{Proving unprovability by de-paradoxizing paradoxes} and (2) \emph{Dynamic modeling of ``platonic'' thought experiments}, $\exists$ntia et Nomin$\forall$ 2011 Workshop,  University of Gda\' nsk, Poland





\vspace{0.5mm}



\emph{A semantical approach to Yablo sentences}, Seventh Barcelona Workshop on Issues in the Theory of Re\-fe\-ren\-ce,  University of Barcelona, Spain


\vspace{0.5mm}


\emph{How not to use the Church-Turing thesis against platonism}, Trends in Logic IX -- Studia Logica International Conference, Church's Thesis: Logic, Mind and Nature,  Jagiellonian University, Cracow, Poland


\vspace{0.5mm}


 \emph{Nominalistic Neologicism},  XX SIUCC SEFA (Sociedad Espa\~ nola de Filosofia Analitica) Workshop with Crispin Wright,    University of  Barcelona, Spain


\vspace{0.5mm}



\years{2010} \emph{Platonism,   Dynamic   Reasoning,   and   Galileo   on   Falling   Bodies}, Logic,   Reasoning,   Rationality,  Ghent   University,
Belgium



\vspace{0.5mm}




  \emph{Nominalism and the modal unstability of Barcan and Reversed Barcan formulas across orders},     CeLL Workshop on Second-Order Modal Logic,  School of Advanced Study, University of London, United Kingdom

\vspace{0.5mm}



  \years{2009} \emph{Anti-Fregean neologicism}, Trends in the Philosophy of Mathematics (Trends in Logic VII),     Goethe University, Frankfurt, Germany


\vspace{0.5mm}


  (1) \emph{Abstraction Principles, a modal interpretation} and (2)   \emph{Platonic  thought   experiments?  Mundane   but  adaptive}, Soci\'et\'e de Philosophie Analytique  congress,
    Geneva, Switzerland


\vspace{0.5mm}



(1)  \emph{Induction from a single instance: incomplete frames} (with Frederik Van De Putte) and (2)   \emph{Adaptive reasoning with dynamic conceptual frames},
  Concept Types and Frames in Language, Cognition, and Science,      Heinrich-Heine-Universit\" at, D\"usseldorf, Germany


\vspace{0.5mm}




 \emph{Plurals and the cumulative set hierarchy},   Non-Classical Mathematics 2009,    Hejnice, Czech Republic


\vspace{0.5mm}


\emph{Swinburne's modal argument for the existence of souls; Formal aspects} (with Agnieszka Rostalska),  Formal Methods in the Epistemology of Religion,   Katholieke Universiteit Leuven, Belgium


\vspace{0.5mm}



%\emph{Abstraction Principles and Existence}, Applications of Logic in Philosophy and Foundations of \mbox{Mathematics},    Szklarska
%Por\c eba, Poland

  \emph{Paradox \& Reference},   PhDs in Logic workshop,    Ghent University, Belgium


\vspace{0.5mm}

%\years{2008} \emph{Ways names could be}, 7th Philosophical Congress,  Warsaw University, Poland




(1) \emph{Logic and Philosophy: a case study: a modal argument for the existence of souls}  (with Agnieszka  Rostalska) and (2) \emph{Capturing dynamic conceptual frames}, Workshop \& Young Researcherís Days in Logic, Philosophy and History of Science,  Palais des Acad\'emies, Brussels, Belgium



%\emph{Le{\'s}niewski's         quantifiers.        A        modal         interpretation},  200         years         of        Analytic         Philosophy,
 %  University of Latvia, Riga, Latvia



\vspace{0.5mm}


 \emph{Reducing sets to modalities}, 31st  International \mbox{Wittgenstein} Symposium of the Austrian Ludwig \mbox{Wittgenstein} Society,  Kirchberg, Austria



\vspace{0.5mm}





%\years{2007}\emph{On a problem with actualism}, commentary at the Calgary Graduate Conference,   University of Calgary, Canada





 \emph{Time Travel and Conditionals},   Logica 2007,   Hejnice, Czech Republic



% \emph{Empty   terms,   square   of   opposition   and   the   practice   of   medieval   logic}, Square   of   Opposition   Congress,  \mbox{Montreux}, Switzerland


\vspace{0.5mm}


  \emph{Tooley's Example and Conditional Logics}, the annual meeting of the Society for Exact Philosophy,      UBC, Vancouver, Canada



\vspace{0.5mm}


 \emph{Some Problems with Le\'sniewski's Foundations of Mathematics},  Mathematical  Methods in Philosophy,  Banff
International Research Station for Mathematical Innovation and Discovery, Canada


\vspace{0.5mm}


\years{2006} \emph{The (in)significance of Tooley's example}, a commentary at the   Western Canadian Philosophical Association Conference,  University of British Columbia, Vancouver, Canada


\vspace{0.5mm}



\vspace{2mm}

\large {\sc \textbf{Editorial \& organizational work}}\normalsize \hspace{5mm}


\normalsize

%\bibstyle{apalike}

\textbf{Associate Editor:} Studia Logica

\textbf{Referee:} Bulletin of Symbolic Logic, Erkenntnis,  Synthese, Council for the Humanities of the Netherlands Organization for Scientific Research, Review of Symbolic Logic,  Studia Logica,  Journal of Philosophical Logic, Journal of Applied Non-Classical Logic,  Logique \& Analyse,   Journal of Philosophical Research,  Foundations of Science,  History and Philosophy of Logic, Theoria, Logic and Logical Philosophy, Polish Journal of Philosophy



\textbf{Main Organizer}: \emph{International Conference for Philosophy of Science and Formal Methods in Philosophy (CoPS-FaM-19) of the Polish Association for Logic and Philosophy of Science}, Gdansk, Poland, \emph{Entia et Nomina 2018}, Gdansk, Poland,  \emph{Entia et Nomina 2017}, Palolem, India, \emph{Swamplandia - Meta-arithmetical results and their philosophical meaning}, Ghent University, 2016, \emph{Academic Caffee of the Polish Academy of Science}, University of Gda{\'n}sk, Poland, 2016, \emph{Trends in Logic XIV}, 2014, Ghent University, Belgium, \emph{Philosophers' Rally}, 2012, University of Gda\' nsk, Poland, \emph{Entia et Nomina 2013, Entia et Nomina 2011, Entia et Nomina 2010}, University of Gda{\'n}sk, Poland,  \emph{Graduate Colloquia Series}, University of Calgary, 2007-2008


%\nobibliography{/Users/rafal/GoogleDrive/Papers/Papers7}


\pagebreak 
\textbf{{\sc \large Leadership}}\\

\vspace{-4mm}


\years{2021} PhD, Micha\l \, Tomasz Godziszewski, Warsaw University, Poland \\ \emph{Semantic Non-conservativeness, non-absoluteness of satisfaction and computable quotient representations of nonstandard models}. (to be defended this year, title tentative)


\years{2020-2021}  Assistant Professor at the Chair, dr. Pavel Janda 

\years{2020-2021} Postdoctoral fellow \\ \emph{Replicability crisis in forensic sciences}, dr Micha\l\, Sikorski

\years{2019-2020} Postdoctoral fellow \\ \emph{Overcoming the paradoxes of naive logical vailidity and informal provablity using non-deterministic truth-maker semantics}, dr Pawe\l\, Paw\l owski


\years{2019-2020} Postdoctoral fellow \\  \emph{Formal Epistemology with Bounded Rationality and Law}, dr. Pavel Janda, University of Gda\'nsk, Poland


\years{2019-2022}  PhD, \emph{Biases in legal applications of machine learning}, Weronika Majek, University of Gda\'nsk, Poland (ongoing, title tentative)



\years{2018} PhD, Pawel Pawlowski, Ghent University, Belgium\\
 \emph{Informally provable, refutable or neither. A non-deterministic approach to informal provability}.

  Committee: Leon Horsten, Hannes Leitgeb, Stewart Shapiro, Frederik Van De Putte, Peter Verd\'ee




%\years{2018-2021} PhD, \emph{Narrative model of legal adjudication \& Bayesianism}, Malgorzata Stefaniak, University of Gda\'nsk, Poland (ongoing)


%\years{2018-2020} MA, Legal dimensions of the reference class problem, Przemek Przepi{\'o}rka, University of Gdansk, Poland (ongoing)

\years{2018-2020} Tutoree accepted for a PhD program in Philosophy at Carnegie Mellon University, Alicja Kowalewska, Poland (ongoing cooperation)

\years{2020-2021} BA supervisee accepted for the MA program in AI at Vrije Universiteit Amsterdam, Alicja Dobrzeniecka

\years{2017-  } Assistant Professor at the Chair, dr. Patryk Dziurosz-Serafinowicz (ongoing cooperation)

\years{2012} MA supervisee accepted for a PhD program at Trinity College Dublin, Zuzanna Gnatek



\vspace{2mm}









\bibliographystyle{plain}


%%\bibliography{/Users/rafal/GoogleDrive/Papers/Papers8,/Users/rafal/GoogleDrive/Papers/Filmat10}
%\bibliography{/Users/rafal/GoogleDrive/Papers/papers15}

\end{document}
